\documentclass[]{article}
\usepackage{lmodern}
\usepackage{amssymb,amsmath}
\usepackage{ifxetex,ifluatex}
\usepackage{fixltx2e} % provides \textsubscript
\ifnum 0\ifxetex 1\fi\ifluatex 1\fi=0 % if pdftex
  \usepackage[T1]{fontenc}
  \usepackage[utf8]{inputenc}
\else % if luatex or xelatex
  \ifxetex
    \usepackage{mathspec}
  \else
    \usepackage{fontspec}
  \fi
  \defaultfontfeatures{Ligatures=TeX,Scale=MatchLowercase}
\fi
% use upquote if available, for straight quotes in verbatim environments
\IfFileExists{upquote.sty}{\usepackage{upquote}}{}
% use microtype if available
\IfFileExists{microtype.sty}{%
\usepackage{microtype}
\UseMicrotypeSet[protrusion]{basicmath} % disable protrusion for tt fonts
}{}
\usepackage[margin=1in]{geometry}
\usepackage{hyperref}
\hypersetup{unicode=true,
            pdftitle={TMA4285 Timeseries, Exercise 3},
            pdfauthor={Marius Dioli, Amir Ahmed and Andreas Ferstad},
            pdfborder={0 0 0},
            breaklinks=true}
\urlstyle{same}  % don't use monospace font for urls
\usepackage{graphicx,grffile}
\makeatletter
\def\maxwidth{\ifdim\Gin@nat@width>\linewidth\linewidth\else\Gin@nat@width\fi}
\def\maxheight{\ifdim\Gin@nat@height>\textheight\textheight\else\Gin@nat@height\fi}
\makeatother
% Scale images if necessary, so that they will not overflow the page
% margins by default, and it is still possible to overwrite the defaults
% using explicit options in \includegraphics[width, height, ...]{}
\setkeys{Gin}{width=\maxwidth,height=\maxheight,keepaspectratio}
\IfFileExists{parskip.sty}{%
\usepackage{parskip}
}{% else
\setlength{\parindent}{0pt}
\setlength{\parskip}{6pt plus 2pt minus 1pt}
}
\setlength{\emergencystretch}{3em}  % prevent overfull lines
\providecommand{\tightlist}{%
  \setlength{\itemsep}{0pt}\setlength{\parskip}{0pt}}
\setcounter{secnumdepth}{0}
% Redefines (sub)paragraphs to behave more like sections
\ifx\paragraph\undefined\else
\let\oldparagraph\paragraph
\renewcommand{\paragraph}[1]{\oldparagraph{#1}\mbox{}}
\fi
\ifx\subparagraph\undefined\else
\let\oldsubparagraph\subparagraph
\renewcommand{\subparagraph}[1]{\oldsubparagraph{#1}\mbox{}}
\fi

%%% Use protect on footnotes to avoid problems with footnotes in titles
\let\rmarkdownfootnote\footnote%
\def\footnote{\protect\rmarkdownfootnote}

%%% Change title format to be more compact
\usepackage{titling}

% Create subtitle command for use in maketitle
\providecommand{\subtitle}[1]{
  \posttitle{
    \begin{center}\large#1\end{center}
    }
}

\setlength{\droptitle}{-2em}

  \title{TMA4285 Timeseries, Exercise 3}
    \pretitle{\vspace{\droptitle}\centering\huge}
  \posttitle{\par}
    \author{Marius Dioli, Amir Ahmed and Andreas Ferstad}
    \preauthor{\centering\large\emph}
  \postauthor{\par}
      \predate{\centering\large\emph}
  \postdate{\par}
    \date{September 2019}

null

\begin{document}
\maketitle

\hypertarget{minimum}{%
\section{Minimum}\label{minimum}}

\begin{enumerate}
\def\labelenumi{\arabic{enumi}.}
\tightlist
\item
  Exploring the data with plotting of relevant statistics
\item
  Transforming the data
\item
  Model parameter estimation including uncertainty
\item
  Diagnostics, and model choice discussion
\end{enumerate}

\hypertarget{abstract}{%
\section{Abstract}\label{abstract}}

\hypertarget{introduction}{%
\section{Introduction}\label{introduction}}

\hypertarget{theory}{%
\section{Theory}\label{theory}}

Outline theory. Start by discussing types of data and behaviour of data.
Then talk about modelling and assumptions. Onto MA and AR, then ARMA,
then ARIMA processes. Estimation and such

\hypertarget{data-analysis}{%
\section{Data Analysis}\label{data-analysis}}

We start by creating simple time series plot to get an overview.
\textless\#Creating basic plots
plot(data.frame(monthly\_flask2\(Date2, monthly_flask2\)CO2\_filled),
type=``o'', ylab=``CO2 Levels'')
plot(data.frame(monthly\_flask1\(Date2, monthly_flask1\)CO2),
type=``o'', ylab=``CO2 Levels'') plot.ts(monthly\_flask2\$CO2\_filled,
type=`s')

\#Averaging the montly reads so that we have one read per month

Just decomposing the time series, we see that there is a clear sesonal
trend, and a clear increasing general trend.

We plot the autoccorlation plot for the time series

Definitely not stationary, apply differencing, to see if it gets any
better.

Now plotting the transformed data we get:

Which could be a stationary process (we use this although there is some
early lag to keep our model less complex and to avoid overfitting).

So we decide on using \(\nabla\nabla^{12}\) as our tranformation.

Which results in a SARIMA model.

(Coefficient can i.e.~be fit using innovations algoritihm. But that
might not be the method impletemented in the package). To find the best
choice of parameters we compare AIC and AICC of different models.

\(p=4, d=1, P=5, Q=1\) seems to be best choice of the remaining
parameters, where we get AIC and AICcc values of 1.27

Fitting the model that had the most optimal values:

The model seem to fit the data quite well.

The standardized residuals seems to be random, and there are not clear
trend. The residuals seems homoscedastic and are centered at zero, which
is a good indcation that our models assumptions are held by the data.

Looking at the autocorrolation plot, there does not seem to be any lag
corrolating that much. Still, there are a few outliers, but that is
often on 12+ months and can be considered to be well within the range of
some family wise error rate.

Points in the qq-plot also seems to align with the normal line quite
well, which indicate a good fit for our model.

\hypertarget{forecasting}{%
\subsection{Forecasting}\label{forecasting}}

We have now fit our model.

The error band in estimation are estimated prediction errors.

Property 3.10 Large Sample Distribution of the Estimators p.~125

\hypertarget{discussion}{%
\section{Discussion}\label{discussion}}

Scoring model using information criteria due to the scarcity of data
(maybe).

Talk about uncertainty, known unknowns, and unknown unknowns.

Weaknesses of chosen model and alternative models.

\hypertarget{conclusion}{%
\section{Conclusion}\label{conclusion}}

\hypertarget{appendix}{%
\section{Appendix}\label{appendix}}

\hypertarget{references}{%
\section{References}\label{references}}


\end{document}
